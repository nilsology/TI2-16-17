\documentclass[
	11pt,								% globale Schriftgröße
	parskip=half-,						% setzt Absatzabstand hoch
	paper=a4,							% Format
	english,ngerman,					% lädt Sprachpakete
	]{scrartcl}							% Dokumentenklasse
	
\input{src/header}						% bindet Header ein (WICHTIG)

% /////////////////////// BEGIN DOKUMENT /////////////////////////
\begin{document}

% /////////////////////// BEGIN TITLEPAGE /////////////////////////
\begin{titlepage}
	\subject{Jochen Schiller}					% <-- Hier Dozenten-Name eintragen
	\title{Rechnerarchitektur}	% <-- Hier Lehrveranst.Name eintragen
	\subtitle{\Large WS 16/17, Übung 1}	% <-- Hier Semester und Nummer eintragen
	\author{%
    	Tutor: Leonard König\\			% <-- Hier Tutor-Namen eintragen
        Tutorium: Donnerstag, 16-18 Uhr\\ \\	% <-- Hier Tutoriumzeit eintragen
        Studierende: Kathrin Le und Nils Koppelmann}		% <-- Hier eure Namen eintragen
	\date{\normalsize \today}					% aktuelles Datum mit \today
\end{titlepage}

\maketitle								% Erstellt das Titelblatt
\vspace*{-11cm}							% rückt Logo an den oberen Seitenrand
\makebox[\dimexpr\textwidth+1cm][r]{	%rechtsbündig und geht rechts 1cm über Layout hinaus
	\includegraphics[width=0.4\textwidth]{src/fu_logo} % fügt FU-Logo ein
}
% /////////////////////// END TITLEPAGE /////////////////////////

\vspace{8.2cm}							% Abstand
\rule{\linewidth}{0.8pt}				% horizontale Linie

% /////////////////////// Aufgabe 1 /////////////////////////
\section*{\Aufgabe{1}{Punkte}}

\subsection*{Interpreter}
Im Vergleich zum Compiler übersetzt der Interpreter den übergebenen Quellcode nicht in ein maschinenlesbares Executable sondern analysiert diesen und führt in aus.\footnote{\url{https://de.wikipedia.org/wiki/Interpreter}}

\subsection*{Compiler}
Ein Compiler übersetzt den übergebenen Quellcode zB. einer Hochsprache in ein maschinenlesbares Executable.\footnote{\url{https://de.wikipedia.org/wiki/Compiler}}

\subsection*{Assembler}
Ein Assembler (Zusammensetzer) übersetzt Assemblersprache in Maschinensprache.\footnote{\url{https://de.wikipedia.org/wiki/Assembler\_(Informatik)}}

\subsection*{Programmiersprache}
Eine Programmiersprache ist ein formales Werkzeug zur Implementierung von Algorithmen und Datenstrukturen. Das Ziel ist einer Rechenmaschiene verständlich Anweisungen zu geben.\footnote{\url{https://wiki.ubuntuusers.de/Programmiersprache/}}

\subsection*{Hochsprache}
Eine Hochsprache ist eine Programmiersprache die nicht hardwarenah ist und entsprechend kompiliert oder interpretiert werden muss. Sie unterscheidet sich sowohl in Abstraktion als auch Komplexität von hardwarenahen Programmiersprachen wie zb. Assemblersprachen.\footnote{\url{https://de.wikipedia.org/wiki/Höhere\_Programmiersprache}}

\subsection*{Assemblersprache}
Eine Assemblersprache ist eine hardwarenahe und oft hardwarespezifische imperative Programmiersprache, die durch einen Assembler kompiliert werden muss.\footnote{\url{https://de.wikipedia.org/wiki/Assemblersprache}}

\subsection*{ISA}
ISA steht für instruction set architecture und erlaubt die Nutzung von Programmen unabhängig von der Implementierung des jeweiligen Mikroprozessors.

\subsection*{CISC}
CISC steht für Complex Instruction Set Computing und beschreibt einen Prozessor mit komplexem Befehlsatz.

\subsection*{RISC}
RISC steht für Reduced Instruction Set Computing und ist im Vergleich zu CISC ein Prozessor mit einem auf die elementaren Befehle beschränkter Befehlsatz.\footnote{\url{http://www.elektronik-kompendium.de/sites/com/0412281.htm}}


% /////////////////////// END DOKUMENT /////////////////////////
\end{document}
